\documentclass[]{article}
\usepackage{lmodern}
\usepackage{amssymb,amsmath}
\usepackage{ifxetex,ifluatex}
\usepackage{fixltx2e} % provides \textsubscript
\ifnum 0\ifxetex 1\fi\ifluatex 1\fi=0 % if pdftex
  \usepackage[T1]{fontenc}
  \usepackage[utf8]{inputenc}
\else % if luatex or xelatex
  \ifxetex
    \usepackage{mathspec}
    \usepackage{xltxtra,xunicode}
  \else
    \usepackage{fontspec}
  \fi
  \defaultfontfeatures{Mapping=tex-text,Scale=MatchLowercase}
  \newcommand{\euro}{€}
\fi
% use upquote if available, for straight quotes in verbatim environments
\IfFileExists{upquote.sty}{\usepackage{upquote}}{}
% use microtype if available
\IfFileExists{microtype.sty}{%
\usepackage{microtype}
\UseMicrotypeSet[protrusion]{basicmath} % disable protrusion for tt fonts
}{}
\usepackage[margin=1in,a4paper]{geometry}
\usepackage{graphicx}
\makeatletter
\def\maxwidth{\ifdim\Gin@nat@width>\linewidth\linewidth\else\Gin@nat@width\fi}
\def\maxheight{\ifdim\Gin@nat@height>\textheight\textheight\else\Gin@nat@height\fi}
\makeatother
% Scale images if necessary, so that they will not overflow the page
% margins by default, and it is still possible to overwrite the defaults
% using explicit options in \includegraphics[width, height, ...]{}
\setkeys{Gin}{width=\maxwidth,height=\maxheight,keepaspectratio}
\ifxetex
  \usepackage[setpagesize=false, % page size defined by xetex
              unicode=false, % unicode breaks when used with xetex
              xetex]{hyperref}
\else
  \usepackage[unicode=true]{hyperref}
\fi
\hypersetup{breaklinks=true,
            bookmarks=true,
            pdfauthor={},
            pdftitle={},
            colorlinks=true,
            citecolor=blue,
            urlcolor=blue,
            linkcolor=magenta,
            pdfborder={0 0 0}}
\urlstyle{same}  % don't use monospace font for urls
\setlength{\parindent}{0pt}
\setlength{\parskip}{6pt plus 2pt minus 1pt}
\setlength{\emergencystretch}{3em}  % prevent overfull lines
\setcounter{secnumdepth}{0}

\usepackage{color}
\usepackage{color}
\usepackage{color}
\date{}

\begin{document}

\section{Response to reviewers}\label{response-to-reviewers}

\section{Reviewer 1}\label{reviewer-1}

\textbf{1. Reviewer's Comment:}

 Page 3, last sentence of the first paragraph in the Introduction is a fragment; consider joining it to
preceding sentence with a colon (:).


\textbf{1. Response:}

\begin{quote}
\color{blue}
We have followed the reviewer's advise.
\end{quote}

\textbf{2. Reviewer's Comment:}

Page 3, last sentence of Introduction: as this is the first mention of BiSSE inbe spelled out (even though it is spelled out in the Abstract as well).


\textbf{2. Response:}

\begin{quote}
\color{blue}
We have followed the reviewer's advise.
\end{quote}

\textbf{3. Reviewer's Comment:}

 Page 3, first paragraph of Data section in Materials \& Methods, final sentence: replace “correct for”
with “account for”.


\textbf{3. Response:}

\begin{quote}
\color{blue}
We have followed the reviewer's advise.
\end{quote}

\textbf{4. Reviewer's Comment:}

Page 3, second paragraph of Data section in Materials \& Methods, first sentence: the two web
resources (Tree of Life Web Project and Global Butterfly Names project) should be listed in the
References section and treated as references in the manuscript (i.e. referred to by number).


\textbf{4. Response:}

\begin{quote}
\color{blue}
We have moved the web links to the references as advised by the reviewer.
\end{quote}

\textbf{5. Reviewer's Comment:}

Page 4, Analyses of Diversification section, second paragraph: The BAMM analyses are given less
than cursory treatment, and no real comparison is discussed in the Discussion (different
diversification rate shifts are identified in the BAMM analyses than in the MultiMEDUSA
analyses). At this point, either a flushed-out discussion of the potential reasons for the differences
should be included, or the BAMM analyses should just be removed from the manuscript (the latter
would likely be best).


\textbf{5. Response:}

\begin{quote}
\color{blue}
We have removed the BAMM analysis from the manuscript.
\end{quote}

\textbf{6. Reviewer's Comment:}

Page 4, Detecting diversification shifts... section: A brief (one to two sentences) description of how
models become iteratively more complex in MEDUSA analyses is warranted here. Something
about how one additional clade-specific diversification rate parameter is added at each step.


\textbf{6. Response:}

\begin{quote}
\color{blue}
We have rewritten the paragraph and included additional sentences as requested
by the reviewer.
\end{quote}


\textbf{7. Reviewer's Comment:}
Page 4, Estimation of trait-dependent... section, first sentence: replace “binary state speciation and extinction” with “BiSSE model”.

\textbf{7. Response:}

\begin{quote}
\color{blue}
We have followed the reviewer's advise.
\end{quote}

\textbf{8. Reviewer's Comment:}

 Page 4, Estimation of trait-dependent... section, third sentence: it is not clear why the reference to MuSSE is necessary here, as the authors do not appear to use it. If this is the case, the sentence could be omitted.
 
 \textbf{8. Response:}

\begin{quote}
\color{blue}
We removed that sentence.
\end{quote}

\textbf{9. Reviewer's Comment:}
Page 5, Estimation of trait-dependent... section, sentence beginning “Other hostplant shifts were not tested...”: is this accurate? At the end of the results section, the very last sentence describes a test for the effect of Apocynaceae feeding on diversification rates...

\textbf{9. Response:}

\begin{quote}
\color{blue}
We removed that sentence.
\end{quote}

\textbf{10. Reviewer's Comment:}

Page 5, Estimation of trait-dependent... section, sentence beginning “We also used constrained...”: it is not immediately clear which model is the constrained model and which model is unconstrained. Start the sentence with: “We also used likelihood ratio tests to evaluate the influence of Solanaceae-feeding on diversification rates by comparing...” and describing the two models being compared.

\textbf{10. Response:}

\begin{quote}
\color{blue}
We corrected the sentence as suggested by the reviewer:

\textbf{We also used likelihood ratio
tests to evaluate the influence of Solanaceae-feeding on diversification rates by comparing a constrained
versus a relaxed model of diversification. The constrained model assumes no effect of hostplant use on
diversification by enforcing equal speciation rates across both character states. In the relaxed model, all
speciation and extinction parameters are estimated for each character state.}
\end{quote}

\textbf{11. Reviewer's Comment:}

Page 5, Estimation of trait-dependent... section, replace “can be feeding on” with “are able to feed on members of”

\textbf{11. Response:}

\begin{quote}
\color{blue}
We made the suggested corrections.
\end{quote}


\textbf{12. Reviewer's Comment:}
 Page 5, Detecting diversification shifts... section: should final number be 1.884, instead of 1,884?

\textbf{12. Response:}

\begin{quote}
\color{blue}
We have removed the whole sentence as the reviewer recommended in point 5.
\end{quote}



\textbf{13. Reviewer's Comment:}
Page 6, sentence beginning “For the diversification shifts..”: the last phrase could use a verb; something like “when the MCC tree alone was used”.

\textbf{13. Response:}

\begin{quote}
\color{blue}
We added "was used" to the end of the sentence.
\end{quote}


\textbf{14. Reviewer's Comment:}
Page 6, sentence beginning “There were four...”: all shifts described are increases. If this is the case, the sentence can be simplified, moving the congruent part of each shift to a single instance preceding the list: “There were four net diversification rate increases found in the trees from the random sample (Table 2); the shifts occurred in: (i) the genus Ypthima (r = 0:22); (ii) the genus Charaxes (r = 0:21); (iii) Ithomiini subtribes Oleriina + Ithomiina + Napeogenina + Dircennina + Godyridina (r = 0:10); and (iv) the clade of Callicore + Diaethria (r = 0:135).”

\textbf{14. Response:}

\begin{quote}
\color{blue}
We modified the sentence as suggested by the reviewer.
\end{quote}



\textbf{15. Reviewer's Comment:}
Page 6, sentence beginning “MultiMEDUSA provided mean and standard deviation...”: This sentence states that “some of the changes...had great variation” - the changes themselves did not have variation, rather, the rates showed considerable variation. Revise sentence to better identify the object that varied.

\textbf{15. Response:}

\begin{quote}
\color{blue}
We modified the sentence to the following:

"MultiMEDUSA provided mean and standard deviation statistics for the
diversification values on the shifts on the 1000 trees (supp. mat.
13--14), and found that the estimated net diversification rate
values had great variation across the posterior distribution of trees."
\end{quote}


\textbf{16. Reviewer's Comment:}
Page 7, final sentence of results: The test of Apocynaceae feeding is not presented in the Methods section.

\textbf{16. Response:}

\begin{quote}
\color{blue}
We added the following sentence to the Methods section:

We also run a BiSSE analysis
to test whether the trait ``feeding on Apocynaceae'' had any effect on
net diversification rates found for Nymphalidae lineages feeding on
Apocynaceae.
\end{quote}


\textbf{17. Reviewer's Comment:}
Page 7, first and third paragraphs of Discussion section (and potentially elsewhere): there are cases where the term “diversification splits” is used, where “diversification shifts” is more appropriate.

\textbf{17. Response:}

\begin{quote}
\color{blue}
We changed ``splits'' to ``shifts'' throughout the text.
\end{quote}


\textbf{18. Reviewer's Comment:}
Page 7, sentence beginning “For example, in our Nymphalidae trees...”: the authors state “we found that the split...had a variation of r = 0.5234”. Should “variation” instead be “rate”?

\textbf{18. Response:}

\begin{quote}
\color{blue}
We corrected the word from ``variation'' to ``rate''.
\end{quote}


\textbf{19. Reviewer's Comment:}
Page 7, third paragraph of Discussion: Replace “This is the reason why” with
“The relatively low phylogenetic support for many nodes in the Nymphalidae tree
is likely the reason why”.

\textbf{19. Response:}

\begin{quote}
\color{blue}
We modified the sentence according to the reviewer's suggestion.
\end{quote}


\textbf{20. Reviewer's Comment:}
Page 7, fourth paragraph of Discussion: Omit this paragraph.

\textbf{20. Response:}

\begin{quote}
\color{blue}
We removed the paragraph according to the reviewer's suggestion.
\end{quote}


\textbf{21. Reviewer's Comment:}
Pages 7-8, fourth paragraph of Discussion: Per previous comment of BAMM analysis, this paragraph should probably be removed. If it is not, both shifts detected by BAMM analysis (not just one) should be described and compared to the four identified in MultiMEDUSA (are they at the same nodes or different nodes?).

\textbf{21. Response:}

\begin{quote}
\color{blue}
We removed the paragraph according to the reviewer's suggestion.
\end{quote}


\textbf{22. Reviewer's Comment:}
Page 8, Hostplant use and diversification... section, Ithomiini subsection,
fifth paragraph, first sentence: insert 'rate': “The increase in diversification
rate inferred by MEDUSA...”

\textbf{22. Response:}

\begin{quote}
\color{blue}
We included the word ``rate'' according to the reviewer's suggestion.
\end{quote}


\textbf{23. Reviewer's Comment:}
Page 8, Hostplant use and diversification... section, Ithomiini subsection,
sixth paragraph, second sentence: replace “might not be too efficient” with 
“may be inefficient”.

\textbf{23. Response:}

\begin{quote}
\color{blue}
We reworded the sentence according to the reviewer's suggestion.
\end{quote}


\textbf{24. Reviewer's Comment:}
Page 8, Hostplant use and diversification... section, Ithomiini subsection,
sixth paragraph, third sentence: make final word plural: “switches”.

\textbf{24. Response:}

\begin{quote}
\color{blue}
We replaced the word ``switch'' according to the reviewer's suggestion.
\end{quote}


\textbf{25. Reviewer's Comment:}
Page 9, first full paragraph, first sentence: replace “hypotheses state” with
“hypotheses posit”.

\textbf{25. Response:}

\begin{quote}
\color{blue}
We replaced the word ``state'' according to the reviewer's suggestion.
\end{quote}


\textbf{26. Reviewer's Comment:}
Page 9, Danaini subsection: Remove the sentence beginning “Among them is for
example...” and just mention the Monarch at the end of the preceding sentence:
“...involved in mimicry rings, including the best-known migratory butterfly,
the monarch (Danaus plexippus).”

\textbf{26. Response:}

\begin{quote}
\color{blue}
We reworded the sentence according to the reviewer's suggestion.
\end{quote}


\end{document}
