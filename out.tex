% !TeX encoding: utf-8
\documentclass[10pt]{article}
\usepackage{amsmath}
\usepackage{amssymb}
\usepackage{cite}
\usepackage{hyperref}
\usepackage{microtype}
\DisableLigatures[f]{encoding = *, family = * }


% Text layout
\topmargin 0.0cm
\oddsidemargin 0.5cm
\evensidemargin 0.5cm
\textwidth 16cm
\textheight 21cm

% Bold the 'Figure #' in the caption and separate it with a period
% Captions will be left justified
\usepackage[labelfont=bf,labelsep=period,justification=raggedright]{caption}

% Use the PLoS provided BiBTeX style
\bibliographystyle{plos2009}

% Remove brackets from numbering in List of References
\makeatletter
\renewcommand{\@biblabel}[1]{\quad#1.}
\makeatother


% Leave date blank
\date{}

\pagestyle{myheadings}

% Include all macros below. Please limit the use of macros.
% END MACROS SECTION
\begin{document}
\begin{flushleft}
{\Large
    \title{Diversity dynamics in Nymphalidae butterflies: Effect of phylogenetic uncertainty on diversification rate shift estimates}
}
\maketitle
Carlos Pe\~na$^{1,\ast}$,
Marianne Espeland$^{2}$
\\
\bf{1} Carlos Pe\~na Laboratory of Genetics, Department of Biology, University of Turku, Turku, Finland. Email: \href{mailto:mycalesis@gmail.com}{\nolinkurl{mycalesis@gmail.com}}
\\
\bf{2} Marianne Espeland Museum of Comparative Zoology and Department
of Organismic and Evolutionary Biology, Harvard University, Cambridge,
USA. Email:
\href{mailto:marianne.espeland@gmail.com}{\nolinkurl{marianne.espeland@gmail.com}}.
Current address: McGuire Center for Lepidoptera and Biodiversity,
Florida Museum of Natural History, University of Florida, Gainesville,
FL, US.
\\
$\ast$ \textbf{Corresponding author:}
\href{mailto:mycalesis@gmail.com}{\nolinkurl{mycalesis@gmail.com}}
Telephone: +358 417063065 Fax: +358 2 333 6680

\end{flushleft}




\subsubsection*{Running title: Diversity dynamics in Nymphalidae
butterflies}


\section*{Abstract}

The species rich butterfly family Nymphalidae has been used to study
evolutionary interactions between plants and insects. Theories of
insect-hostplant dynamics predict accelerated diversification due to key
innovations. In evolutionary biology, analysis of maximum credibility
trees in the software MEDUSA (modelling evolutionary diversity using
stepwise AIC) is a popular method for estimation of shifts in
diversification rates. We investigated whether phylogenetic uncertainty
can produce different results by extending the method across a random
sample of trees from the posterior distribution of a Bayesian run. Using
the MultiMEDUSA approach, we found that phylogenetic uncertainty greatly
affects diversification rate estimates. Different trees produced
diversification rates ranging from high values to almost zero for the
same clade, and both significant rate increase and decrease in some
clades. Only four out of 18 significant shifts found on the maximum
clade credibility tree were consistent across most of the sampled trees.
Among these, we found accelerated diversification for Ithomiini
butterflies. We used the binary speciation and extinction model (BiSSE)
and found that a hostplant shift to Solanaceae is correlated with
increased net diversification rates in Ithomiini, congruent with the
diffuse cospeciation hypothesis. Our results show that taking
phylogenetic uncertainty into account when estimating net
diversification rate shifts is of great importance, as very different
results can be obtained when using the maximum clade credibility tree
and other trees from the posterior distribution.

\textbf{Keywords:} diversification analysis, MEDUSA, BiSSE, speciation
rate, insect-hostplant dynamics

\section*{Introduction}

Hostplant shifts have been invoked to be responsible for a great part of
the biodiversity of herbivorous insects {[}1{]}. The study of the
evolution of hostplant use has spawned several theories explaining the
evolutionary interactions between plants and insects {[}2{]}. The
``escape-and-radiate'' hypothesis {[}3{]}, the ``oscillation
hypothesis'' {[}4,5{]} or ``diffuse cospeciation'' {[}2{]}.

The butterfly family Nymphalidae has been an important taxon for
developing some of the mentioned hypotheses. Nymphalidae contains around
6000 species {[}6{]}, and several members are considered model organisms
in evolutionary biology {[}7--9{]}. The family most likely originated
around 94 MYA in the mid Cretaceous. Diversification of the group began
in the Late Cretaceous and most major radiations (current subfamilies)
appeared shortly after the Cretaceous-Paleogene (K-Pg) boundary
{[}10{]}. Several studies have used time-calibrated phylogenies and
diversification models to reconstruct the evolutionary history of the
group to identify patterns of accelerated or decelerated diversification
of some Nymphalidae clades {[}10--13{]}. For example, it has been
suggested that climate change in the Oligocene and the subsequent
diversification of grasses has led to diversification of the subfamily
Satyrinae {[}14{]} due to the abundance of grasses over extensive
geographic areas (``resource abundance-dependent diversity dynamics''
hypothesis). Fordyce (2010) {[}12{]} found increased net diversification
rates in some Nymphalidae lineages after a major hostplant shift, which
appears to be in agreement with the ``escape-and-radiate'' model of
diversification {[}3{]}.

Although it has been suggested that part of the great diversity of
Nymphalidae butterflies is a result of hostplant-insect dynamics, it is
necessary to use modern techniques to investigate whether the
diversification patterns of Nymphalidae are in agreement with the
theoretical predictions. It is necessary to test whether the overall
diversification pattern of Nymphalidae is congruent with events of
sudden diversification bursts due to hostplant shift, climatic events or
shifts to closely related hostplants {[}5,15{]}.

In this study, we used a time-calibrated genus-level phylogenetic
hypothesis for Nymphalidae butterflies {[}13{]} to investigate patterns
of diversification. We applied MEDUSA {[}16,17{]}, a recently developed
statistical method, to study the diversification pattern of Nymphalidae
butterflies. MEDUSA fits likelihood models of diversification into a
time-calibrated tree and tests whether allowing increases or decreases
in speciation and extinction rates within the tree produces better fit
of the models. MEDUSA is able to take into account unsampled extant
species diversity during model fitting and it is normally applied to the
maximum clade credibility phylogenetic tree. Particularly, we wanted to
study the effects of phylogenetic uncertainty and by using the extended
MEDUSA method called MultiMEDUSA {[}16{]}. We also tested whether
hostplant association dynamics can explain the diversification patterns
of component Nymphalidae lineages by testing whether character states of
hostplant use affected the diversification pattern of those lineages
employing the method BiSSE as implemented in the R package
\texttt{diversitree} {[}18{]}.

\section*{Materials and Methods}

\subsection*{Data}

For analyses, we used the phylogenetic trees from the study of Wahlberg
et al. (2009) {[}13{]} that were generated using DNA sequence data from
10 gene regions for 398 of the 540 valid genera in Nymphalidae. We
employed the maximum clade credibility tree (MCC tree) (Fig. 1) as well
as a random sample of 1000 trees from their BEAST run after burnin
{[}13{]}. Their original BEAST run was for 40 million generations. We
used a burnin of 25 million generations and took a random sample of 1000
trees using Burntrees v.0.1.9 \url{http://www.abc.se/~nylander/} (supp.
mat. 5) in order to correct for phylogenetic uncertainty when performing
the diversification analyses.

Species richness data for Nymphalidae genera were compiled from several
sources including the specialist-curated lists on
\url{http://tolweb.org}, Lamas (2004) {[}19{]} and curated lists of
Global Butterfly Names project (\url{http://www.ucl.ac.uk/taxome/gbn/}).
We assigned the species numbers of genera not included in the phylogeny
to the closest related genus that was included in Wahlberg et al. (2009)
{[}13{]} study according to available phylogenetic studies {[}20--31{]},
taxonomical classification and morphological resemblance when no
phylogenies were available (supp. mat. 3).

Hostplant data for Nymphalidae species were compiled from several
sources including {[}32{]}, HOSTS database (\url{http://bit.ly/YI7nwW}),
{[}33--35{]} and others (supp. mat. 15--16) for a total of 6586
hostplant records, including 428 Nymphalidae genera and 143 plant
families and 1070 plant genera. It was not possible to find any
hostplant data for 35 butterfly genera.

\subsection*{Analyses of Diversification}

We used the statistical software R version 3.0.1 {[}36{]} in combination
with the APE {[}37{]}, GEIGER {[}38{]}, MEDUSA {[}16{]} and
\texttt{diversitree} {[}18{]} packages along with our own scripts to
perform the analyses (included as supplementary materials 2, 6, 10). All
analyses were run on the 1000 random trees from Wahlberg et al. (2009)
{[}13{]}, on the maximum clade credibility tree derived from these and
the MCC tree {[}13{]}.

\subsection*{Detecting diversification shifts on phylogenetic
trees}

Patterns of diversification in Nymphalidae were analyzed by using MEDUSA
version 093.4.33 {[}16{]} on the maximum clade credibility tree from
Wahlberg et al. (2009) {[}13{]}. MEDUSA fits alternative birth-death
likelihood models to a phylogenetic tree in order to estimate changes in
net diversification rates along branches. MEDUSA estimates likelihood
and AIC scores for the simplest birth-death model, with two parameters,
the rates \texttt{r}: net diversification \texttt{=} speciation
(\texttt{b}) \texttt{-} extinction (\texttt{d}) and ($\varepsilon$):
relative extinction \texttt{= d/b}. The AIC scores of the two-parameter
model are then compared with incrementally more complex models until the
addition of parameters do not improve the AIC scores beyond a cutoff
value. MEDUSA finds the likelihood of the models after taking into
account branch lengths and number of species per lineage {[}16{]}. Most
studies using MEDUSA only run the method on a single tree, usually the
maximum clade credibility tree, which makes the assumption that this
tree is correct. We wanted to study the effects of phylogenetic
uncertainty on estimation of net diversification rate shifts and
therefore selected 1000 random genus-level trees from the posterior
distribution of the Bayesian run from Wahlberg et al., 2009 {[}13{]}
(MultiMEDUSA, supp. mat. 5). We calculated a new MCC tree from the
selection of trees, ran MEDUSA and MultiMEDUSA on the selection of 1000
trees and summarized the estimated changes in net diversification rates
for nodes across all trees. Patterns of change in net diversification
rates are considered significant if they are found at the same node in
at least 90\% of the trees. We also expected to find similar values of
net diversification (\texttt{r = b - d}) and relative extinction rate
($\varepsilon$) \texttt{= d/b}) values across the 1000 trees for the
nodes where changes in diversification tempo occurs. We used the AICc
threshold of 7.8 units, as estimated by MEDUSA, as the limit for a
significantly better fit to select among increasingly complex
alternative models.

\subsection*{Estimation of trait-dependent speciation rates}

As the MEDUSA and MultiMEDUSA approaches estimated an increase in net
diversification in the clade Ithomiini, we tested whether this pattern
can be explained by an increase in the birth-rate due to hostplant use
and performed analyses using the ``binary state speciation and
extinction'' {[}39{]} Bayesian approach as implemented in the R package
\texttt{diversitree} {[}18{]}. The Markov Chain Monte Carlo algorithm
was run for 10000 generations discarding the first 7500 as burnin. The
Multiple State Speciation and Extinction approach (MuSSE) {[}18{]} is
designed to examine the joint effects of two or more traits on
speciation. Because most of Nymphalidae butterflies are restricted to
use one plant family as hostplant (232 genera use only one plant family
versus 176 that use more than one; based on our data in supp. mat. 15),
the character states can be coded as presence/absence, for which the
BiSSE analysis is better suited. BiSSE was designed to test whether a
binary character state has had any effect on increased net
diversification rate for a clade {[}39{]}. We used our compiled data of
hostplant use to produce a binary dataset for the character ``feeding on
the plant family Solanaceae'' with two states (absence = 0; presence =
1) (supp. mat. 17). The family Solanaceae include the hostplants of the
diverse Ithomiini butterflies and closest relatives {[}8{]}. Other
hostplant shifts were not tested for effect on net diversification rates
due to the low support for diversification shifts found in the MEDUSA
analyses. We analyzed the data using BiSSE employing the Markov Chain
Monte Carlo algorithm on the maximum clade credibility tree, taking into
account missing taxa by using the parameter ``sampling factor''
(\texttt{sampling.f}) in \texttt{diversitree}. We also used constrained
analyses forcing no effect of hostplant use on diversification and used
likelihood ratio tests to find out whether the hypothesis of effect on
diversification has a significantly better likelihood than the null
hypothesis (no effect). The analyses were run across a sample of 250
trees from the selected 1000 random trees from the posterior
distribution. The records of \emph{Vanessa} and \emph{Hypanartia}
feeding on Solanaceae {[}33,40{]} might be incorrect as it is unlikely
that these species can be feeding on this plant family. We tested
whether coding these two genera as not feeding on Solanaceae affected
our results. We expected that coding these two genera as either absence
or presence would not have a significant effect on our results. It has
been shown that BiSSE performs poorly under certain conditions {[}41{]}.
However, our data has adequate number of taxa under analysis (more than
300 tips), adequate speciation bias (between 1.5x and 2.0x), character
state bias (around 8x) and extinction bias (around 4x) for the analysis
of Solanaceae hostplants. Thus, BiSSE is expected to produce robust
results {[}41{]}.

\section*{Results}

\subsection*{Detecting diversification shifts on the maximum clade
credibility tree}

The MEDUSA analysis on the MCC tree in combination with richness data
estimated 18 changes in the tempo of diversification in Nymphalidae
history (Fig. 1; Table 1). The corrected AICc acceptance threshold for
adding subsequent piecewise birth-death processes to the overall model
was set to 7.8 units, as prescribed by MEDUSA. In all MEDUSA analyses,
the maximum number of inferred diversification splits in all trees was
26. The background net diversification rate for Nymphalidae was
estimated as \texttt{r = 0.092} lineages per Million of years and the
AICc score for the best fit model was \texttt{5090.5} (Table 1). MEDUSA
also estimated that the basic constant birth-death model was not a
better explanation for our data (AICc \texttt{= 5449.3}).

Some of the 18 changes in diversification correspond to rate increases
in very species-rich genera: \emph{Ypthima} (\texttt{r = 0.311}),
\emph{Charaxes} (\texttt{r = 0.291}), \emph{Callicore} +
\emph{Diaethria} (\texttt{r = 0.220}), \emph{Pedaliodes}
(\texttt{r = 0.196}) and \emph{Taenaris} (\texttt{r = 0.234}). We found
rate increases for other clades as well such as: Mycalesina
(\texttt{r = 0.191}), Oleriina + Ithomiina + Napeogenina + Dircennina +
Godyridina (\texttt{r = 0.187}), Satyrini (\texttt{r = 0.116}),
Phyciodina in part (\texttt{r = 0.241}) and Satyrina
(\texttt{r = 0.221}), \emph{Coenonympha} (\texttt{r = 0.209}),
\emph{Caeruleuptychia} + \emph{Magneuptychia} (\texttt{r = 0.312}) and
\emph{Taenaris} (\texttt{r = 0.312}). We also found decreases in net
diversification rates for Limenitidinae + Heliconiinae
(\texttt{r = 0.0541}), part of Danaini (\texttt{r = 0.0423}),
Pseudergolinae (\texttt{r = 0.024}) and Coenonymphina
(\texttt{r = 0.065}) (Table 1).

We contrasted these results with those obtained from analysing the data in
the software BAMM {[}42{]}. We found
that the best model (best configuration) estimated 2 diversification shifts in
our MCC tree (posterior probability = 0.79) for the Ithomiini
 and Satyrini. Bayes factor (BF) values strongly 
favored this model in comparison with a constant rate model (no diversification
shifts, BF $>$ 1,884).

\subsection*{Phylogenetic uncertainty in the MultiMEDUSA approach}

We used MEDUSA to find out whether taking into account the phylogenetic
signal from the random sample of 1000 trees from the posterior
distribution can return similar estimates of diversification to the
values obtained from the MCC tree. We ran MultiMEDUSA on the random
sample of 1000 trees (supp. mat. 5) from the posterior distribution and
compared the results with a MEDUSA analysis on the MCC tree derived from
this sample (supp. mat. 9).

We found that the analysis by MultiMEDUSA on the 1000 trees estimated
lower median net diversification rates for the diversification shifts
found by MEDUSA on the MCC tree from {[}13{]} (Table 2). Although the
diversification pattern found by MEDUSA and MultiMEDUSA was the same,
the latter consistently estimated lower rates. Furthermore, the shifts
recovered with low net diversification rate on the MCC were recovered
with negative net diversification rate by MultiMEDUSA. The background
diversification and all shifts found by MEDUSA on the 1000 trees are
provided as an R object in supp. mat. 7.

We also compared the results from MultiMEDUSA (derived from the sample
of 1000 trees) with the splits found by MEDUSA on the MCC tree derived
from this random sample. In the summary statistics, MultiMEDUSA reports
the frequency of the diversification shifts found in the trees
(parameter \texttt{sum.prop}). Thus, if a node is found in only half of
the 1000 trees, but the phylogenetic signal was strong enough to be
picked up by MEDUSA and a node shift was found most of the time, then
the \texttt{sum.prop} should be close to 1. For example, the
\texttt{Charaxes} + \texttt{Polyura} clade was found in only 256 trees,
however MultiMEDUSA was consistently able to find a diversification
shift for that node and the \texttt{sum.prop} value is 0.996.

For the diversification shifts found in both the MCC tree and most of
the samples of 1000 trees (frequency more than 90\%; Table 2), the
MultiMEDUSA approach recovered different rates of diversification than
those found using when the MCC tree alone.

There were four diversification shifts found with more than 0.90 of
probability in the trees from the random sample (Table 2): (i) net
diversification rate increase in the genus \emph{Ypthima}; (ii) net
diversification rate increase in the genus \emph{Charaxes}; (iii) rate
increase in Ithomiini subtribes Oleriina + Ithomiina + Napeogenina +
Dircennina + Godyridina; and (iv) rate increase in \emph{Callicore} +
\emph{Diaethria}.

MultiMEDUSA provided mean and standard deviation statistics for the
diversification values found on the shifts on the 1000 trees (supp. mat.
13--14), and found that some of the changes in net diversification rate
values had great variation across the posterior distribution of trees. A
boxplot of the net diversification rate values estimated for the clades
that appear in the MCC tree shows that some shifts are estimated as
increased or slowed diversification pace depending on the tree used for
analysis (Fig. 2). This variation is especially wide for the clade
formed by the genera \emph{Magneuptychia} and \emph{Caeruleuptychia}
because MEDUSA estimated diversification values from six times the
background net diversification rate (\texttt{r = 0.5234}) to almost zero
(\texttt{r = 1.22e-01}). The rates for \emph{Taenaris} were between 0.14
and 0.44 (mean value 0.25). Similar degrees of variation were found in
the nodes for \emph{Ypthima}, \emph{Charaxes} and \emph{Coenonympha}
(Fig. 2). The net diversification rates estimates for the clades
(Oleriina + Ithomiina + Napeogenina + Dircennina + Godyridina),
Limenitidinae + Heliconiinae and Pseudergolinae are relatively
consistent across the 1000 trees (Fig. 2).

It is also evident that not all the diversification shifts estimated on
the MCC tree are consistently recovered in most of the 1000 trees. Some
of the splits in the MCC tree are recovered in very few trees, for
example the split for the clade Satyrini (Euptychiina + Pronophilina +
Satyrina + Maniolina) is recovered with a probability of 0.18 (Fig. 3).

\subsection*{Estimation of trait-dependent speciation rates}

The MEDUSA analyses, taking into account phylogenetic uncertainty,
estimated a net diversification rate increase in part of the clade
Ithomiini across more than 95\% of the trees. Our BiSSE analysis found a
positive effect of the character state ``feeding on Solanaceae'' on the
net net diversification rate on part of Ithomiini (Oleriina + Ithomiina
+ Napeogenina + Dircennina + Godyridina) (Fig. 4). The estimated mean
net diversification rate for taxa that do not feed on Solanaceae was
\texttt{r = 0.11} while the net diversification rate for the Solanaceae
feeders was \texttt{r = 0.16} (see Fig. S1 for a boxplot of speciation
and extinction values for the 95\% credibility intervals). The same
analysis considering \emph{Vanessa} and \emph{Hypanartia} as
non-Solanaceae feeders due to dubious records produced the same pattern
and net diversification rates (Fig. S2). Therefore, the rest of the
analyses were performed assuming these two genera as Solanaceae feeders.
We constrained the BiSSE likelihood model to force equal rates of
speciation for both character states in order to test whether the model
of different speciation rates is a significantly better explanation for
the data. A likelihood ratio test found that the model for increased net
diversification rate for nymphalids feeding on Solanaceae is a
significantly better explanation than this character state having no
effect on diversification ($\chi^2 = 12.3; 1 df; p < 0.001$) (Table 3;
character states available in supp. mat. 17, code in supp. mat. 18, and
mcmc run in supp. mat. 19). We combined the post-burnin mcmc generations
from running BiSSE on 250 trees from the random sample of 1000 trees and
found the same pattern as the BiSSE analysis on the maximum clade
credibility tree (combined mcmc run in supp. mat. 20; profiles plot of
speciation rates in Fig. S3; boxplot of 95\% credibility intervals in
Fig. S4). A BiSSE analysis to test whether the trait ``feeding on
Apocynaceae'' had any effect on increased net diversification rates
found similar speciation rates for lineages feeding on Apocynaceae and
other plants (Fig. S5).

\section*{Discussion}

\subsection*{Effects of phylogenetic uncertainty on the performance of MEDUSA}

The MEDUSA method has been used to infer changes in net diversification
rates in a phylogenetic tree. Since its publication {[}16{]} the results
of using MEDUSA on a single tree, the maximum clade credibility tree,
have been used for generation of hypotheses and discussion
{[}10,43,44{]}. However, different diversification shifts and different
rates of diversification are found for certain lineages when
phylogenetic uncertainty was taken into account by using MEDUSA on a
random sample of trees from the posterior distribution of a Bayesian
run. We found that some diversification splits, estimated on the
Nymphalidae maximum clade credibility tree, were found with very low
probability in the random sample of 1000 trees from the posterior
distribution (Fig. 3, Table 2). We also found that, even though the
analyses estimated the same diversification splits on two or more trees,
the estimated net diversification rates could vary widely (Fig. 2). For
example, in our Nymphalidae trees, we found that the split for
\emph{Magneuptychia} and \emph{Caeruleuptychia} had a variation from
\texttt{r = 0.5234}, higher than the background net diversification
rate, to almost zero. This means that observed patterns and conclusions
can be completely contradictory depending on tree choice.

In this study, the effect of phylogenetic uncertainty on the inferred
diversification splits by MEDUSA is amplified because some Nymphalidae
taxa appear to be strongly affected by long-branch attraction artifacts
{[}30{]}. Thus, the Bayesian runs are expected to recover alternative
topologies on the posterior distribution of trees, resulting in low
support and posterior probability values for the nodes. For example,
posterior probability values for clades in Satyrini are very low
{[}13{]}. As a result, MEDUSA inferred a net diversification rate
increase for part of Satyrini in the maximum clade credibility tree, but
this was recovered in the MultiMEDUSA analysis in only in 13\% of the
random sample of trees.

If there is strong phylogenetic signal for increases or decreases in net
diversification rates for a node, it is expected that these splits would
be inferred by MEDUSA in most of the posterior distribution of trees.
However, weak phylogenetic signal for some nodes can cause some clades
to be absent in some trees and MEDUSA will be unable to estimate any
diversification shift (due to a non-existent node). This is the reason
why MEDUSA estimated net diversification rate splits with a probability
higher than 0.90 in the sample of trees for only four splits: the genus
\emph{Charaxes}, the genus \emph{Ypthima}, part of Ithomiini and the
clade \emph{Callicore} + \emph{Diaethria} (Fig. 3), while estimating
splits for other lineages with much lower probability.

The clade Ithomiini and the non-basal danaids are well supported by high
posterior probability values in Wahlberg et al. (2009) {[}13{]}.
Therefore our MEDUSA analyses recovered an increase in net
diversification rate with probability higher than 0.90 in the posterior
distribution of trees (Fig. 3).

\subsection*{Hostplant use and diversification in Nymphalidae}

\subsubsection*{Ithomiini}

Keith Brown suggested that feeding on Solanaceae was an important event
in the diversification of Ithomiini butterflies {[}45{]}. Ithomiini
butterflies are exclusively Neotropical and most species feed on
Solanaceae hostplants during larval stage {[}8{]}. Optimizations of the
evolution of hostplant use on phylogenies evidence a probable shift from
Apocynaceae to Solanaceae in the ancestor of the tribe {[}8,46{]}.
Fordyce (2010) {[}12{]} found that the Gamma statistics, a LTT plot of
an Ithomiini phylogeny and the fit of the density-dependent model of
diversification are consistent with a burst of diversification in
Ithomiini following the shift from Apocynaceae to Solanaceae.

We investigated whether the strong signal for an increase in net
diversification rate for Ithomiini (found by MEDUSA) can be explained
due to the use of Solanaceae plants as hosts during larval stage. For
this, we used a Bayesian approach {[}47{]} to test whether the trait
``feeding on Solanaceae'' had any effect on the diversification of the
group.

Our BiSSE analysis, extended to take into account missing taxa and
phylogenetic uncertainty, shows a significantly higher net
diversification rate for Ithomiini taxa, which can be attributed to the
trait ``feeding on Solanaceae hostplants'' (Fig. 4). This is in
agreement with previoues findings using other statistical methods
{[}12{]}. Due to the fact that Ithomiini are virtually the only
nymphalids using Solanaceae as hostplants, it is possible that the trait
responsible for a higher diversification of Ithomiini might not be the
hostplant character. As noted by Maddison et al. (2007) {[}39{]}, the
responsible trait might be a codistributed character such as a trait
related to the ability to digest secondary metabolites.

Solanaceae plants contain chemical compounds and it has been suggested
that the high diversity of Ithomiini is consistent with the
``escape-and-radiate scenario'' due to a shift onto Solanaceae {[}12{]}
and radiation scenarios among chemically different lineages of
Solanaceae plants {[}8,45{]}. According to this theory, the shift from
Apocynaceae to Solanaceae allowed Ithomiini to invade newly available
resources due to a possible key innovation that allowed them cope with
secondary metabolites of the new hosts. Additional studies are needed to
identify the actual enzymes that Ithomiini species might be using for
detoxification of ingested food as they have been found in other
butterfly groups {[}48{]}.

The increase in diversification inferred by MEDUSA ocurred after the
probable shift from Apocynaceae to Solanaceae, as the Solanaceae feeders
in the subtribes Melinaeina and Mechanitina are not included in the
diversification shift (shift number 5 in Fig. 1). The apparent
conflicting results from MEDUSA and BiSSE can be explained by the low
species-richness of the subtribes Melinaeina and Mechanitina compared to
the other subtribes included in the shift (52 versus 272 species). It
can be that MEDUSA is more conservative than BiSSE and is not including
Melinaeina and Mechanitina in the shift due to low species numbers.

Although the Solanaceae genera used by the Ithomiini clades are well
known {[}8{]}, we do not have any understanding on the physiological
routes involved in the detoxification of Solanaceae compounds by the
several lineages of Ithomiini. We can speculate that older lineages
exploiting a novel toxic resource {[}8,13{]} might not be too efficient
in metabolizing plant toxins and that younger lineages are able to deal
with toxins more efficiently, so that host switching events within
Solanaceae are possible, which can lead to higher diversification.
Studies in \emph{Papilio} species have reported that detoxification
enzymes can become more efficient in metabolizing toxins than ancestral
configurations of the proteins, providing more opportunities for
hostplant switch {[}49{]}. This might be the reason why the basal
Ithomiini subtribes Melinaeina and Mechanitina are so species-poor and
restricted to few Solanaceae hosts {[}8{]}, while recent subtribes are
species-rich and have expanded their host range into several Solanaceae
lineages {[}8{]}. It might be that the switch to feeding in Solanaceae
was an important event in the evolutionary history of Ithomiini, but the
actual radiation occurred after critical physiological changes (a
probable key innovation) allowed efficient detoxification of Solanaceae
toxins.

The diffuse cospeciation hypothesis predicts almost identical ages of
insects and their hostplants, while the ``resource abundance-dependent
diversity'' and the ``escape-and-radiate'' hypotheses state that insects
diversify after their hostplants {[}2--4{]}. Wheat et al. (2007)
{[}48{]} found strong evidence for a model of speciation congruent with
Ehrlich and Raven's hypothesis in Pieridae butterflies due to, in
addition to the identification of a key innovation, a burst of
diversification in glucosinolate-feeding taxa shortly afterwards (with a
lag of circa 10 MY). According to a recent dated phylogeny of
the Angiosperms {[}50{]}, the family Solanaceae split from its sister
group about 59 (49--68) MYA and diversification started (crown group
age) around 37 (29--47) MYA. Wahlberg et al. (2009) {[}13{]} give the
ages for origin and diversification for Ithomiini at 45 (39--53) and 37
(32--43) MYA, respectively. Thus, current evidence shows that Solanaceae
and Ithomiini might have diversified around the same time, during the
Late Eocene and Oligocene, and this would be congruent with the diffuse
cospeciation hypothesis.

\subsubsection*{Danaini}
Our MultiMEDUSA approach showed a significant slowdown in net
diversification rate in the subtribe Danaina of the Danini. Both Danaina
and the sister clade Euploeina feed mainly on Apocynaceae and thus a
hostplant shift should not be responsible for the observed slowdown of
diversification in the Danaina. As expected, our BiSSE analysis of
Apocynaceae feeders shows that there is no effect of feeding on this
plant family on the net diversification rates of Nymphalidae lineages.
Many of the Danaina are large, strong fliers, highly migratory and
involved in mimicry rings. Among them is for example the monarch
(\emph{Danaus plexippus}), probably the most well known of all migratory
butterflies. The causes for a lower net diversification rate in the
Danaina remains to be investigated, but their great dispersal power
might be involved in preventing allopatric speciation. It has been found
in highly vagile species in the nymphalid genus \emph{Vanessa} that
dispersal has homogenized populations due to gene flow, as old and
vagile species seem to be genetically homogeneous while younger
widespread species show higher genetic differentiation in their
populations {[}51{]}.

\subsubsection*{Satyrini}

The diverse tribe Satyrini radiated simultaneously with the radiation of
their main hostplant, grasses, during the climatic cooling in the
Oligocene {[}14{]}. Thus, it is somewhat surprising that part of
Satyrini (the subtribes Euptychiina, Satyrina and Pronophilina) were
found to have accelerated diversification in only 13\% of the trees from
the posterior distribution. Although this can be attributed to low
phylogenetic signal {[}13{]}, the clade Satyrini is very robust {[}13{]}
and MEDUSA failed to identify any significant accelerated net
diversification rate for Satyrini. It appears that the radiation of
Satyrini as a whole was not remarkably fast and therefore not picked up
by MEDUSA. This should be expected if the diversification of Satyrinae
occurred in a stepwise manner, with pulses or bursts of diversification
for certain lineages but unlikely for the tribe Satyrini as a whole.

\section*{Conclusions}
We found that even though MEDUSA estimated several diversification
shifts in the maximum clade credibility tree of Nymphalidae, only a few
of these splits were found in more than 90\% of the trees from the
posterior distribution. In the literature, it is common practice that
conclusions are based on the splits estimated on the maximum clade
credibility tree. However, by using a MultiMEDUSA approach, we found
that for this Nymphalidae dataset some of these splits might be greatly
affected by phylogenetic uncertainty. Moreover, some of these splits can
be recovered either as increases or decreases in net diversification
rate depending on the tree from the posterior distribution that was used
for analysis. This means that contradictory conclusions would be made if
only the maximum clade credibility tree was used for analysis. We
recommend that all datasets should be analyzed using both approaches,
MEDUSA and MultiMEDUSA, in order to test whether the results are robust
when phylogenetic uncertainty is taken into account.

MEDUSA appears to be sensitive to the number of nodes with high
posterior probability and width of age confidence intervals. For our
data, it would be necessary to obtain a posterior distribution of trees
with no conflicting topology, and very similar estimated ages for nodes
in order to consistently recover most of the diversification splits on
the posterior distribution of trees that were inferred by MEDUSA on the
MCC tree.

Our MultiMEDUSA approach to perform analyses on the posterior
distribution of trees found strong support for an increase in net
diversification rate in the tribe Ithomiini and the genus
\emph{Charaxes}, and for a decrease in net diversification rate in the
subtribe Danaina. Due to phylogenetic uncertainty, we did not obtain
strong support for other diversification splits in Nymphalidae. Our
BiSSE analysis corroborated other studies in that the trait 
``feeding on Solanaceae'', or a codistributed character, was important in the
diversification of Ithomiini butterflies. However, by applying MEDUSA we
found that a critical character in the radiation of the group might have
appeared after the shift from Apocynaceae to Solanaceae. We also found
that the trait ``feeding on Apocynaceae'' is not responsible for the
slowdown of diversification in Danaina. Ithomiini and Solanaceae
diversified near simultaneously, which is in agreement with the diffuse
co-speciation hypothesis {[}2,4{]}.

\section*{Acknowledgments}
We are thankful to Mark Cornwall for help with the script to extend
MEDUSA to include phylogenetic uncertainty, Niklas Wahlberg for
commenting on the manuscript and giving us the posterior distribution of
trees, Luke Harmon for commenting on the manuscript and anonymous
reviewers for their comments, which greatly improved the manuscript,
Jessica Slove Davidson and Niklas Janz for access to their hostplant
data. The study was supported by a Kone Foundation grant (awarded to
Niklas Wahlberg), Finland (C. Pe\~na) and the Research Council of Norway
(grant no. 204308 to M. Espeland). We acknowledge CSC--IT Center for
Science Ltd. (Finland) for the allocation of computational resources.

\section*{Figure legends}
\begin{description}
 \item {\bf Figure 1. Results of the MEDUSA analysis run on the maximum
clade credibility tree from Wahlberg et al. (2009)} Rate shifts were estimated for the
following nodes (besides the background rate): 2) Limenitidinae +
Heliconiinae, 3) \emph{Ypthima}, 4) \emph{Charaxes}, 5) Ithomiini in
part, 6) Satyrini, 7) Coenonymphina 8) Phyciodina in part, 9) Danaini in
part, 10) \emph{Coenonympha}, 11) \emph{Caeruleuptychia} +
\emph{Magneuptychia}, 12) \emph{Callicore} + \emph{Diaethria}, 13)
Satyrina, 14) Mycalesina, 15) \emph{Pedaliodes}, 16) \emph{Dryas} +
\emph{Dryadula}, 17) \emph{Taenaris}, 18) Pseudergolinae. Circles on
nodes indicate the diversification shift number as found by MEDUSA.
Numbers next to circles indicat the posterior probability values for
such nodes.

\item {\bf Figure 2. Diversification rates for taxa estimated by MEDUSA on
the samples of 1000 random trees}

\item {\bf Figure 3. Results of MultiMEDUSA analysis showing the
probability of specific nodes being characterized by significant shifts
in diversification rate}

\item {\bf Figure 4. BiSSE analysis of diversification of nymphalids due
to feeding on Solanaceae hostplants} Speciation and net diversification
rates are significantly higher in Solanaceae feeders (speciation rate =
$\lambda1$, net diversification rate = $r1$.
\end{description}


\section*{Supporting Information Legends}
\begin{description}
 \item {\bf Supporting Information S01.nex.} MCC Nymphalidae tree from
{[}13{]}

 \item {\bf Supporting Information S02.R} R script to run MEDUSA on the
MCC Nymphalidae tree from {[}13{]}. This script removes the outgroup
taxa and loads the richness data for the tree terminals.

 \item {\bf Supporting Information S03.csv} Species richness for lineages
in Nymphalidae.

 \item {\bf Supporting Information S04.txt} Results of running MEDUSA on
the MCC tree.

 \item {\bf Supporting Information S05 1000 random trees no outgroups.zip} 1000 random trees from {[}13{]}.

 \item {\bf Supporting Information S06.R}?R script to run a MultiMEDUSA
analysis on 1000 random trees from {[}13{]}.

 \item {\bf Supporting Information S07 multimedusa on 1000 trees.zip} Raw results from the MultiMEDUSA run on the random sample of trees from
the posterior distribution.

 \item {\bf Supporting Information S08.pdf} Summary of the MulitMEDUSA
analysis on the 1000 trees from the posterior distribution.

 \item {\bf Supporting Information S09.tree} MCC tree from the 1000
random trees selected from the posterior distribution.

 \item {\bf Supporting Information S10.R} R script to run a MEDUSA
analysis on the MCC tree from the 1000 random trees selected from the
posterior distribution.

 \item {\bf Supporting Information S11.txt} Summary results from a MEDUSA
run on the MCC tree from the 1000 random trees selected from the
posterior distribution.

 \item {\bf Supporting Information S12.pdf} Figure for MEDUSA run on MCC
tree from random 1000 trees.

 \item {\bf Supporting Information S13.txt} Summary results from a
MultiMEDUSA run on the 1000 random trees selected from the posterior
distribution.

 \item {\bf Supporting Information S14.pdf} Probability of nodes with
estimated rates from a MultiMEDUSA run on the 1000 random trees selected
from the posterior distribution.

 \item {\bf Supporting Information S15.csv} Hostplants of Nymphalidae
butterflies recorded from the literature.

 \item {\bf Supporting Information S16.csv} References for hostplants
data.

 \item {\bf Supporting Information S17.csv} Data matrix with character
states for hosplant use.

 \item {\bf Supporting Information S18.R} R script for running the BiSSE
analysis.

 \item {\bf Supporting Information S19.csv} Raw results for the BiSSE
analysis.

 \item {\bf Supporting Information S20.csv} Raw results for the combined
BiSSE analysis.

 \item {\bf Figure S1.} Boxplot of speciation ($\lambda$) and extinction
($\mu$) values for the 95\% credibility intervals of values estimated
by BiSSE analysis of diversification due to feeding on Solanaceae
plants.

 \item {\bf Figure S2.} BiSSE analysis of diversification of nymphalids due
to feeding on Solanaceae hostplants assuming \emph{Vanessa} and
\emph{Hypanartia} as non-Solanaceae feeders. The same pattern is
recovered, speciation and net diversification rates are significantly
higher for Solanaceae feeders $\lambda1$, r1).

 \item {\bf Figure S3.} Net diversification rates of nymphalids feeding on
Solanaceae plants as estimated by combining post-burnin runs of BiSSE on
the 1000 trees from the posterior distribution.

 \item {\bf Figure S4.} Boxplot of speciation and extinction values for the
95\% credibility intervals of values estimated by BiSSE analysis of
diversification due to feeding on Solanaceae plants on the combined
post-burnin runs on 1000 trees from the posterior distribution.

 \item {\bf Figure S5.} BiSSE analysis of diversification of nymphalids due
to feeding on Apocynaceae hostplants. Speciation and net diversification
rates are similar.
\end{description}




\section*{References}
1. Mitter C, Farrell B, Wiegmann B (1988) The phylogenetic study of
adaptive zones: Has phytophagy promoted insect diversification? Am Nat
132: 107--128.
doi:\href{http://dx.doi.org/10.1086/284840}{10.1086/284840}.

2. Nyman T, Linder HP, Pe{\~n}a C, Malm T, Wahlberg N (2012) Climate-driven
diversity dynamics in plants and plant-feeding insects. Ecol Lett 15:
889--898.
doi:\href{http://dx.doi.org/10.1111/j.1461-0248.2012.01782.x}{10.1111/j.1461-0248.2012.01782.x}.

3. Ehrlich PR, Raven PH (1964) Butterflies and plants: A study in
coevolution. Evolution 18: 586--608.
doi:\href{http://dx.doi.org/10.2307/2406212}{10.2307/2406212}.

4. Janz N (2011) Ehrlich and Raven Revisited: Mechanisms Underlying
Codiversification of Plants and Enemies. Annu Rev Ecol Evol S 42:
71--89.
doi:\href{http://dx.doi.org/10.1146/annurev-ecolsys-102710-145024}{10.1146/annurev-ecolsys-102710-145024}.

5. Nylin S, Slove J, Janz N (2014) Host plant utilization, host range
oscillations and diversification in Nymphalid butterflies: A
phylogenetic investigation. Evolution 68: 105--124.
doi:\href{http://dx.doi.org/10.1111/evo.12227}{10.1111/evo.12227}.

6. Nieukerken EJ van, Kaila L, Kitching IJ, Kristensen NP, Lees DC, et
al. (2011) Order Lepidoptera. In: Animal biodiversity: An outline of
higher-level classification and survey of taxonomic richness. Zootaxa.
pp. 212--221.

7. Joron M, Jiggins C, Papanicolaou A, McMillan W (2006) Heliconius wing
patterns: An evo-devo model for understanding phenotypic diversity.
Heredity 97: 157--167.

8. Willmott KR, Freitas AVL (2006) Higher-level phylogeny of the
Ithomiinae (Lepidoptera: Nymphalidae): classification, patterns of
larval hostplant colonization and diversification. Cladistics 22:
297--368.
doi:\href{http://dx.doi.org/10.1111/j.1096-0031.2006.00108.x}{10.1111/j.1096-0031.2006.00108.x}.

9. Brakefield PM, Beldade P, Zwaan BJ (2009) The african butterfly
bicyclus anynana: A model for evolutionary genetics and evolutionary
developmental biology. Cold Spring Harbor Protocols 2009: pdb--emo122.

10. Heikkil{\"a} M, Kaila L, Mutanen M, Pe{\~n}a C, Wahlberg N (2012) Cretaceous
origin and repeated tertiary diversification of the redefined
butterflies. P Roy Soc B 279: 1093--1099.
doi:\href{http://dx.doi.org/10.1098/rspb.2011.1430}{10.1098/rspb.2011.1430}.

11. Elias M, Joron M, Willmott KR, Silva-Brand{\~a}o KL, Kaiser V, et al.
(2009) Out of the Andes: Patterns of diversification in clearwing
butterflies. Mol Ecol 18: 1716--1729.
doi:\href{http://dx.doi.org/10.1111/j.1365-294X.2009.04149.x}{10.1111/j.1365-294X.2009.04149.x}.

12. Fordyce JA (2010) Host shifts and evolutionary radiations of
butterflies. P Roy Soc B 277: 3735--3743.
doi:\href{http://dx.doi.org/10.1098/rspb.2010.0211}{10.1098/rspb.2010.0211}.

13. Wahlberg N, Leneveu J, Kodandaramaiah U, Pe{\~n}a C, Nylin S, et al.
(2009) Nymphalid butterflies diversify following near demise at the
Cretaceous/Tertiary boundary. P Roy Soc B 276: 4295--4302.
doi:\href{http://dx.doi.org/10.1098/rspb.2009.1303}{10.1098/rspb.2009.1303}.

14. Pe{\~n}a C, Wahlberg N (2008) Prehistorical climate change increased
diversification of a group of butterflies. Biol Lett 4: 274--278.
doi:\href{http://dx.doi.org/10.1098/rsbl.2008.0062}{10.1098/rsbl.2008.0062}.

15. Ferrer-Paris JR, S{\'a}nchez-Mercado A, Viloria {\'A}L, Donaldson J (2013)
Congruence and Diversity of Butterfly-Host Plant Associations at Higher
Taxonomic Levels. PloS one 8: e63570.
doi:\href{http://dx.doi.org/0.1371/journal.pone.0063570}{0.1371/journal.pone.0063570}.

16. Alfaro ME, Santini F, Brock C, Alamillo H, Dornburg A, et al. (2009)
Nine exceptional radiations plus high turnover explain species diversity
in jawed vertebrates. P Natl Acad Sci USA 106: 13410--13414.
doi:\href{http://dx.doi.org/10.1073/pnas.0811087106}{10.1073/pnas.0811087106}.

17. Harmon LJ, Rabosky DL, {FitzJohn} RG, Brown JW (2011) turboMEDUSA:
MEDUSA: Modeling Evolutionary Diversification Using Stepwise AIC.

18. {FitzJohn} RG (2012) Diversitree: comparative phylogenetic analyses
of diversification in R. Method Ecol Evol 3: 1084--1092.
doi:\href{http://dx.doi.org/10.1111/j.2041-210X.2012.00234.x}{10.1111/j.2041-210X.2012.00234.x}.

19. Lamas G (2004) Checklist: Part 4A. Hesperioidea-Papilionoidea.
Gainesville.

20. Matos-Marav{\'i} PF, Pe{\~n}a C, Willmott KR, Freitas AV, Wahlberg N (2013)
Systematics and evolutionary history of butterflies in the ``Taygetis
clade'' (Nymphalidae: Satyrinae: Euptychiina): Towards a better
understanding of Neotropical biogeography. Mol Phylogenet Evol 66:
54--68.
doi:\href{http://dx.doi.org/10.1016/j.ympev.2012.09.005}{10.1016/j.ympev.2012.09.005}.

21. Brower AVZ, Wahlberg N, Ogawa JR, Boppr{\'e} M, Vane-Wright RI (2010)
Phylogenetic relationships among genera of danaine butterflies
(Lepidoptera: Nymphalidae) as implied by morphology and DNA sequences.
Syst Biodivers 8: 75--89.
doi:\href{http://dx.doi.org/10.1080/14772001003626814}{10.1080/14772001003626814}.

22. Kodandaramaiah U, Lees DC, M{\"u}ller CJ, Torres E, Karanth KP, et al.
(2010) Phylogenetics and biogeography of a spectacular Old World
radiation of butterflies: the subtribe Mycalesina (Lepidoptera:
Nymphalidae: Satyrini). BMC Evol Biol 10: 172.
doi:\href{http://dx.doi.org/10.1186/1471-2148-10-172}{10.1186/1471-2148-10-172}.

23. Kodandaramaiah U, Pe{\~n}a C, Braby MF, Grund R, M{\"u}ller CJ, et al.
(2010) Phylogenetics of Coenonymphina (Nymphalidae: Satyrinae) and the
problem of rooting rapid radiations. Mol Phylogenet Evol 54: 386--394.
doi:\href{http://dx.doi.org/10.1016/j.ympev.2009.08.012}{10.1016/j.ympev.2009.08.012}.

24. Ortiz-Acevedo E, Willmott KR (2013) Molecular systematics of the
butterfly tribe Preponini (Nymphalidae: Charaxinae). Syst Entomol 38:
440--449.
doi:\href{http://dx.doi.org/10.1111/syen.12008}{10.1111/syen.12008}.

25. De-Silva DL, Day JJ, Elias M, Willmott KR, Whinnett A, et al. (2010)
Molecular phylogenetics of the Neotropical butterfly subtribe Oleriina
(Nymphalidae: Danainae: Ithomiini). Mol Phylogenet Evol 55: 1032--1041.
doi:\href{http://dx.doi.org/10.1016/j.ympev.2010.01.010}{10.1016/j.ympev.2010.01.010}.

26. Freitas AVL, Brown Jr. KS (2004) Phylogeny of the Nymphalidae
(Lepidoptera). Syst Biol 53: 363--383.
doi:\href{http://dx.doi.org/10.1080/10635150490445670}{10.1080/10635150490445670}.

27. Pe{\~n}a C, Wahlberg N, Weingartner E, Kodandaramaiah U, Nylin S, et al.
(2006) Higher level phylogeny of Satyrinae butterflies (Lepidoptera:
Nymphalidae) based on DNA sequence data. Mol Phylogenet Evol 40: 29--49.
doi:\href{http://dx.doi.org/10.1016/j.ympev.2006.02.007}{10.1016/j.ympev.2006.02.007}.

28. Penz CM (1999) Higher level phylogeny for the passion-vine
butterflies (Nymphalidae, Heliconiinae) based on early stage and adult
morphology. Zool J Linn Soc 127: 277--344.
doi:\href{http://dx.doi.org/10.1111/j.1096-3642.1999.tb00680.x}{10.1111/j.1096-3642.1999.tb00680.x}.

29. Silva-Brand{\~a}o KL, Wahlberg N, Francini RB, Azeredo-Espin AML, Brown
Jr. KS, et al. (2008) Phylogenetic relationships of butterflies of the
tribe Acraeini (Lepidoptera, Nymphalidae, Heliconiinae) and the
evolution of host plant use. Mol Phylogenet Evol 46: 515--531.
doi:\href{http://dx.doi.org/10.1016/j.ympev.2007.11.024}{10.1016/j.ympev.2007.11.024}.

30. Pe{\~n}a C, Nylin S, Wahlberg N (2011) The radiation of Satyrini
butterflies (Nymphalidae: Satyrinae): a challenge for phylogenetic
methods. Zool J Linn Soc 161: 64--87.
doi:\href{http://dx.doi.org/10.1111/j.1096-3642.2009.00627.x}{10.1111/j.1096-3642.2009.00627.x}.

31. Pe{\~n}a C, Nylin S, Freitas AVL, Wahlberg N (2010) Biogeographic
history of the butterfly subtribe Euptychiina (Lepidoptera, Nymphalidae,
Satyrinae). Zool Scr 39: 243--258.
doi:\href{http://dx.doi.org/10.1111/j.1463-6409.2010.00421.x}{10.1111/j.1463-6409.2010.00421.x}.

32. Ackery PR (1988) Hostplants and classification: A review of
nymphalid butterflies. Biol J Linn Soc 33: 95--203.
doi:\href{http://dx.doi.org/10.1111/j.1095-8312.1988.tb00446.x}{10.1111/j.1095-8312.1988.tb00446.x}.

33. Dyer LA, Gentry GL (2002) Caterpillars and parasitoids of a tropical
lowland wet forest. Available: \url{http://www.caterpillars.org}.
Accessed 17 April 2013.

34. Beccaloni G, Viloria {\'A}, Hall S, Robinson G (2008) Catalogue of the
hostplants of the Neotropical butterflies. London: The Natural History
Museum.

35. Janzen DH, Hallwachs W (2009) Dynamic database for an inventory of
the macrocaterpillar fauna, and its food plants and parasitoids, of Area
de Conservacion Guanacaste (ACG), northwestern Costa Rica. Available:
\url{http://janzen.sas.upenn.edu}. Accessed 17 April 2013.

36. Core Team R (2013) R: A Language and Environment for Statistical
Computing. Vienna, Austria: R Foundation for Statistical Computing.
Available: \url{http://www.r-project.org}.

37. Popescu A-A, Huber KT, Paradis E (2012) ape 3.0: New tools for
distance-based phylogenetics and evolutionary analysis in R.
Bioinformatics 28: 1536--1537.
doi:\href{http://dx.doi.org/10.1093/bioinformatics/bts184}{10.1093/bioinformatics/bts184}.

38. Harmon LJ, Weir JT, Brock CD, Glor RE, Challenger W (2008) GEIGER:
investigating evolutionary radiations. Bioinformatics 24: 129--131.
doi:\href{http://dx.doi.org/10.1093/bioinformatics/btm538}{10.1093/bioinformatics/btm538}.

39. Maddison WP, Midford PE, Otto SP (2007) Estimating a binary
character's effect on speciation and extinction. Syst Biol 56: 701--710.
doi:\href{http://dx.doi.org/10.1080/10635150701607033}{10.1080/10635150701607033}.

40. Scott JA (1986) The Butterflies of North America. Stanford
University Press.

41. Davis MP, Midford PE, Maddison W (2013) Exploring power and
parameter estimation of the BiSSE method for analyzing species
diversification. BMC Evol Biol 13: 38.
doi:\href{http://dx.doi.org/10.1186/1471-2148-13-38}{10.1186/1471-2148-13-38}.

42. Rabosky DL (2014) Automatic Detection of Key Innovations, Rate Shifts, and
Diversity-Dependence on Phylogenetic Trees. PLoS ONE 9: e89543. doi:
\href{http://dx.doi.org/10.1371/journal.pone.0089543}{10.1371/journal.pone.0089543}.

43. Litman JR, Danforth BN, Eardley CD, Praz CJ (2011) Why do leafcutter
bees cut leaves? New insights into the early evolution of bees. P Roy
Soc B 278: 3593--3600.
doi:\href{http://dx.doi.org/10.1098/rspb.2011.0365}{10.1098/rspb.2011.0365}.

44. Ryberg M, Matheny PB (2012) Asynchronous origins of ectomycorrhizal
clades of Agaricales. P Roy Soc B 279: 2003--2011.
doi:\href{http://dx.doi.org/10.1098/rspb.2011.2428}{10.1098/rspb.2011.2428}.

45. Brown Jr. KS (1987) Chemistry at the Solanaceae/Ithomiinae
Interface. Ann Mo Bot Gard 74: 359--397.

46. Brower AVZ, Freitas AVL, Lee M-M, Silva-Brand{\~a}o KL, Whinnett A, et
al. (2006) Phylogenetic relationships among the Ithomiini (Lepidoptera:
Nymphalidae) inferred from one mitochondrial and two nuclear gene
regions. Syst Entomol 31: 288--301.
doi:\href{http://dx.doi.org/10.1111/j.1365-3113.2006.00321.x}{10.1111/j.1365-3113.2006.00321.x}.

47. {FitzJohn} RG, Maddison WP, Otto SP (2009) Estimating
trait-dependent speciation and extinction rates from incompletely
resolved phylogenies. Syst Biol 58: 595--611.
doi:\href{http://dx.doi.org/10.1093/sysbio/syp067}{10.1093/sysbio/syp067}.

48. Wheat CW, Vogel H, Wittstock U, Braby MB, Underwood D, et al. (2007)
The genetic basis of a plant-insect coevolutionary key innovation. P
Natl Acad Sci USA 104: 20427--20431.
doi:\href{http://dx.doi.org/10.1073/pnas.0706229104}{10.1073/pnas.0706229104}.

49. Li W, Schuler MA, Berenbaum MR (2003) Diversification of
furanocoumarin-metabolizing cytochrome P450 monooxygenases in two
papilionids: Specificity and substrate encounter rate. P Natl Acad Sci
USA 100: 14593--14598.
doi:\href{http://dx.doi.org/10.1073/pnas.1934643100}{10.1073/pnas.1934643100}.

50. Bell CD, Soltis DE, Soltis PS (2010) The age and diversification of
the angiosperms re-revisited. Am J Bot 97: 1296--1303.
doi:\href{http://dx.doi.org/10.3732/ajb.0900346}{10.3732/ajb.0900346}.

51. Wahlberg N, Rubinoff D (2011) Vagility across \emph{Vanessa} (Lepidoptera:
Nymphalidae): mobility in butterfly species does not inhibit the
formation and persistence of isolated sister taxa. Syst Entomol 36:
362--370.
doi:\href{http://dx.doi.org/10.1111/j.1365-3113.2010.00566.x}{10.1111/j.1365-3113.2010.00566.x}.


% Table 1
\section*{Tables}
\begin{table}[!h]
\caption{\bf{Significant net diversification rate shifts found in the MEDUSA analysis of the Nymphalid maximum clade credibility tree.}}
\begin{tabular}{lccrrcl}
Shift N$^\circ$ & Split.Node & Model & r          & LnLik.part & AICc     & Taxa                                                       \\
1               & 399        & yule  & 0.092459   & -1055.957  &          & Nymphalidae (root)                                         \\
2               & 691        & bd    & 0.054129   & -406.3703  &          & Limenitidinae + Heliconiinae                               \\
3               & 299        & yule  & 0.311199   & -6.3058    &          & \emph{Ypthima}                                             \\
4               & 224        & yule  & 0.290989   & -6.2601    &          & \emph{Charaxes}                                            \\
5               & 750        & yule  & 0.186913   & -147.4146  &          & Oleriina + Ithomiina + Napeogenina + Dircennina + Godyrina \\
6               & 405        & yule  & 0.116252   & -555.0276  &          & Satyrini                                                   \\
7               & 495        & yule  & 0.064656   & -124.9143  &          & Coenonymphina                                              \\
8               & 609        & yule  & 0.240562   & -35.0908   &          & Phyciodina in part                                         \\
9               & 787        & bd    & 0.042332   & -43.7819   &          & Danaini in part                                            \\
10              & 231        & yule  & 0.209416   & -4.7955    &          & \emph{Coenonympha}                                         \\
11              & 478        & yule  & 0.311684   & -9.2218    &          & \emph{Caeruleuptychia} + \emph{Magneuptychia}              \\
12              & 659        & yule  & 0.219253   & -11.2686   &          & \emph{Callicore} + \emph{Diaethria}                        \\
13              & 444        & yule  & 0.220615   & -43.7812   &          & Satyrina                                                   \\
14              & 524        & yule  & 0.190754   & -26.0651   &          & Mycalesina                                                 \\
15              & 355        & yule  & 0.234041   & -6.2013    &          & \emph{Pedaliodes}                                          \\
16              & 714        & yule  & 0          & 0          &          & \emph{Dryas} + \emph{Dryadula}                             \\
17              & 377        & yule  & 0.311671   & -4.1986    &          & \emph{Taenaris}                                            \\
18              & 688        & yule  & 0.024724   & -17.5256   &          & Pseudergolinae                                             \\
19              & 583        & yule  & 0          & 0          & 5090.492 & \emph{Anaeomorpha} + \emph{Hypna}
\end{tabular}
\begin{flushleft}Split.Node=node number, Model=preferred diversification model by MEDUSA, r=net diversification rate, LnLik.part=log likelihood value.
\end{flushleft}
\end{table}


% Table 2
\begin{table}[!h]
    \caption{\bf{Differences in rates estimated by MEDUSA on the MCC
tree from Wahlberg and the MultiMEDUSA approach on 1000 random trees
from the posterior distribution.}}
\begin{tabular}{lcrr}
Split.Node & rate by MEDUSA & Median rate by MultiMEDUSA & probability of being recovered \\
1          & 0.090          & not found                  & 0.000                          \\
2          & 0.054          & -0.030                     & 0.864                          \\
3          & 0.311          & 0.219                      & 0.961                          \\
4          & 0.290          & 0.212                      & 0.996                          \\
5          & 0.187          & 0.101                      & 0.927                          \\
6          & 0.116          & 0.039                      & 0.131                          \\
7          & 0.064          & -0.052                     & 0.195                          \\
8          & 0.240          & 0.166                      & 0.319                          \\
9          & 0.042          & -0.049                     & 0.897                          \\
10         & 0.209          & 0.149                      & 0.619                          \\
11         & 0.311          & 0.208                      & 0.831                          \\
12         & 0.219          & 0.135                      & 0.911                          \\
13         & 0.220          & 0.117                      & 0.276                          \\
14         & 0.190          & 0.115                      & 0.379                          \\
15         & 0.234          & 0.127                      & 0.825                          \\
16         & 0.000          & -0.005                     & 0.659                          \\
17         & 0.311          & 0.248                      & 0.785                          \\
18         & 0.024          & -0.064                     & 0.024                          \\
19         & 0.000          & -0.087                     & 0.381
\end{tabular}
\end{table}


% Table 3
\begin{table}[!h]
    \caption{\bf{Likelihood ratio test between the model of increased
diversification of nymphalids feeding on Solanaceae against a model
forcing equal speciation rates (no effect on diversification).}}
\begin{tabular}{lcrrlr}
Df           & lnLik & AIC     & ChiSq  & \multicolumn{1}{r}{p} &         \\
full         & 6     & -1613.3 & 3238.5 &                       &         \\
equal.lambda & 5     & -1619.4 & 3248.9 & 12.3                  & 0.00045
\end{tabular}
\end{table}




\end{document}
